\section{Embedded Linux system development - part 2}

\begin{frame}
  \frametitle{Contents}
  \begin{itemize}
  \item Using open-source components
  \item Tools for the target device
    \begin{itemize}
    \item Networking
    \item System utilities
    \item Language interpreters
    \item Audio, video and multimedia
    \item Graphical toolkits
    \item Databases
    \item Web browsers
    \end{itemize}
  \item \st{System building}
  \item Commercial tool sets and distributions
  \end{itemize}
\end{frame}

\subsection{Leveraging open-source components in an Embedded Linux
  system}

\begin{frame}
  \frametitle{Third party libraries and applications}
  \begin{itemize}
  \item One of the advantages of embedded Linux is the wide range of
    third-party libraries and applications that one can leverage in
    its product
    \begin{itemize}
    \item They are freely available, freely distributable, and thanks
      to their open-source nature, they can be analyzed and modified
      according to the needs of the project
    \end{itemize}
  \item However, efficiently re-using these components is not always
    easy. One must:
    \begin{itemize}
    \item Find these components
    \item Choose the most appropriate ones
    \item Cross-compile them
    \item Integrate them in the embedded system and with the other
      applications
    \end{itemize}
  \end{itemize}
\end{frame}

\begin{frame}
  \frametitle{Find existing components}
  \begin{itemize}
  \item Free Software Directory\\
    \url{http://directory.fsf.org}
  \item Look at other embedded Linux products, and see what their
    components are
  \item Look at the list of software packaged by embedded Linux build
    systems
    \begin{itemize}
    \item These are typically chosen for their suitability to embedded
      systems
    \end{itemize}
  \item Ask the community or Google
  \item This presentation will also feature a list of components for
    common needs
  \end{itemize}
\end{frame}

\begin{frame}
  \frametitle{Choosing components}
  \begin{itemize}
  \item Not all free software components are necessarily good to
    re-use. One must pay attention to:
    \begin{itemize}
    \item {\bf Vitality} of the developer and user communities. This
      vitality ensures long-term maintenance of the component, and
      relatively good support. It can be measured by looking at the
      mailing-list traffic and the version control system activity.
    \item {\bf Quality} of the component. Typically, if a component is
      already available through embedded build systems, and has a
      dynamic user community, it probably means that the quality is
      relatively good.
    \item {\bf License}. The license of the component must match your
      licensing constraints. For example, GPL libraries cannot be used
      in proprietary applications.
    \item {\bf Technical requirements}. Of course, the component must
      match your technical requirements. But don't forget that you can
      improve the existing components if a feature is missing!
    \end{itemize}
  \end{itemize}
\end{frame}

\begin{frame}
  \frametitle{Licenses (1)}
  \begin{itemize}
  \item All software that are under a free software license give four
    freedoms to all users
    \begin{itemize}
    \item Freedom to use
    \item Freedom to study
    \item Freedom to copy
    \item Freedom to modify and distribute modified copies
    \end{itemize}
  \item See \url{http://www.gnu.org/philosophy/free-sw.html} for a
    definition of Free Software
  \item Open Source software, as per the definition of the Open Source
    Initiative, are technically similar to Free Software in terms of
    freedoms
  \item See \url{http://www.opensource.org/docs/osd} for the definition of
    Open Source Software
  \end{itemize}
\end{frame}

\begin{frame}
  \frametitle{Licenses (2)}
  \begin{itemize}
  \item Free Software licenses fall in two main categories
    \begin{itemize}
    \item The copyleft licenses
    \item The non-copyleft licenses
    \end{itemize}
  \item The concept of {\em copyleft} is to ask for reciprocity in the
    freedoms given to a user.
  \item The result is that when you receive a software under a
    copyleft free software license and distribute modified versions of
    it, you must do so under the same license
    \begin{itemize}
    \item Same freedoms to the new users
    \item It's an incentive to contribute back your changes instead of
      keeping them secret
    \end{itemize}
  \item Non-copyleft licenses have no such requirements, and modified
    versions can be kept proprietary, but they still require
    attribution
  \end{itemize}
\end{frame}

\begin{frame}
  \frametitle{GPL}
  \begin{itemize}
  \item {\bf GNU General Public License}
  \item Covers around 55\% of the free software projects
    \begin{itemize}
    \item Including the Linux kernel, Busybox and many applications
    \end{itemize}
  \item Is a copyleft license
    \begin{itemize}
    \item Requires derivative works to be released under the same
      license
    \item Programs linked with a library released under the GPL must
      also be released under the GPL
    \end{itemize}
  \item Some programs covered by version 2 (Linux kernel, Busybox and
    others)
  \item More and more programs covered by version 3, released in 2007
    \begin{itemize}
    \item Major change for the embedded market: the requirement that
      the user must be able to {\bf run} the modified versions on the
      device, if the device is a {\em consumer} device
    \end{itemize}
  \end{itemize}
\end{frame}

\begin{frame}
  \frametitle{GPL: redistribution}
  \begin{itemize}
  \item No obligation when the software is not distributed
    \begin{itemize}
    \item You can keep your modifications secret until the product
      delivery
    \end{itemize}
  \item It is then authorized to distribute binary versions, if one of
    the following conditions is met:
    \begin{itemize}
    \item Convey the binary with a copy of the source on a physical
      medium
    \item Convey the binary with a written offer valid for 3 years
      that indicates how to fetch the source code
    \item Convey the binary with the network address of a location
      where the source code can be found
    \item See section 6. of the GPL license
    \end{itemize}
  \item In all cases, the attribution and the license must be
    preserved
    \begin{itemize}
    \item See section 4. and 5.
    \end{itemize}
  \end{itemize}
\end{frame}

\begin{frame}
  \frametitle{LGPL}
  \begin{itemize}
  \item {\bf GNU Lesser General Public License}
  \item Covers around 10\% of the free software projects
  \item A copyleft license
    \begin{itemize}
    \item Modified versions must be released under the same license
    \item But, programs linked against a library under the LGPL do not
      need to be released under the LGPL and can be kept proprietary.
    \item However, the user must keep the ability to update the
      library independently from the program. Dynamic linking is the
      easiest solution. Statically linked executables are only possible
      if the developer provides a way to relink with an update
      (with source code or linkable object files).
    \end{itemize}
  \item Used instead of the GPL for most of the libraries, including
    the C libraries
    \begin{itemize}
    \item Some exceptions: MySQL, or Qt \textless = 4.4
    \end{itemize}
  \item Also available in two versions, v2 and v3
  \end{itemize}
\end{frame}

\begin{frame}
  \frametitle{Licensing: examples}
  \begin{itemize}
  \item You make modifications to the Linux kernel (to add drivers or
    adapt to your board), to Busybox, U-Boot or other GPL software
    \begin{itemize}
    \item You must release the modified versions under the same
      license, and be ready to distribute the source code to your
      customers
    \end{itemize}
  \item You make modifications to the C library or any other LGPL
    library
    \begin{itemize}
    \item You must release the modified versions under the same
      license
    \end{itemize}
  \item You create an application that relies on LGPL libraries
    \begin{itemize}
    \item You can keep your application proprietary, but you must link
      dynamically with the LGPL libraries
    \end{itemize}
  \item You make modifications to a non-copyleft licensed software
    \begin{itemize}
    \item You can keep your modifications proprietary, but you must
      still credit the authors
    \end{itemize}
  \end{itemize}
\end{frame}

\begin{frame}
  \frametitle{Non-copyleft licenses}
  \begin{itemize}
  \item A large family of non-copyleft licenses that are relatively
    similar in their requirements
  \item A few examples
    \begin{itemize}
    \item Apache license (around 4\%)
    \item BSD license (around 6\%)
    \item MIT license (around 4\%)
    \item X11 license
    \item Artistic license (around 9 \%)
    \end{itemize}
  \end{itemize}
\end{frame}

\begin{frame}[fragile]
  \frametitle{BSD license}
  \begin{block}{}
  \tiny
  \begin{verbatim}
Copyright (c) <year>, <copyright holder>
All rights reserved.

Redistribution and use in source and binary forms, with or without
modification, are permitted provided that the following conditions are
met:
* Redistributions of source code must retain the above copyright
  notice, this list of conditions and the following disclaimer.
* Redistributions in binary form must reproduce the above copyright
  notice, this list of conditions and the following disclaimer in the
  documentation and/or other materials provided with the distribution.
* Neither the name of the <organization> nor the names of its
  contributors may be used to endorse or promote products derived from
  this software without specific prior written permission.

[...]
\end{verbatim}
  \end{block}
\end{frame}

\begin{frame}
  \frametitle{Is this free software?}
  \begin{itemize}
  \item Most of the free software projects are covered by ~10
    well-known licenses, so it is fairly easy for the majority of
    project to get a good understanding of the license
  \item Otherwise, read the license text
  \item Check Free Software Foundation's opinion\\
    \url{http://www.fsf.org/licensing/licenses/}
  \item Check Open Source Initiative's opinion\\
    \url{http://www.opensource.org/licenses}
  \end{itemize}
\end{frame}

\begin{frame}
  \frametitle{Respect free software licenses}
  \begin{itemize}
  \item Free Software is not public domain software, the distributors
    have obligations due to the licenses
    \begin{itemize}
    \item {\bf Before} using a free software component, make sure the
      license matches your project constraints
    \item Make sure to keep a complete list of the free software
      packages you use, the original version you used and to keep your
      modifications and adaptations well-separated from the original
      version
    \item Conform to the license requirements before shipping the
      product to the customers
    \end{itemize}
  \item Free Software licenses have been enforced successfully in
    courts. Organizations which can help:
    \begin{itemize}
    \item GPL-violations.org, \url{http://www.gpl-violations.org}
    \item Software Freedom Law Center, \url{http://www.softwarefreedom.org/}
    \item Software Freedom Conservancy, \url{http://sfconservancy.org/}
    \end{itemize}
  \item Ask your legal department!
  \end{itemize}
\end{frame}

\begin{frame}
  \frametitle{Keeping changes separate (1)}
  \begin{itemize}
  \item When integrating existing open-source components in your
    project, it is sometimes needed to make modifications to them
    \begin{itemize}
    \item Better integration, reduced footprint, bug fixes, new
      features, etc.
    \end{itemize}
  \item Instead of mixing these changes, it is much better to keep
    them separate from the original component version
    \begin{itemize}
    \item If the component needs to be upgraded, easier to know what
      modifications were made to the component
    \item If support from the community is requested, important to
      know how different the component we're using is from the
      upstream version
    \item Makes contributing the changes back to the community
      possible
    \end{itemize}
  \item It is even better to keep the various changes made on a given
    component separate
    \begin{itemize}
    \item Easier to review and to update to newer versions
    \end{itemize}
  \end{itemize}
\end{frame}

\begin{frame}
  \frametitle{Keeping changes separate (2)}
  \begin{itemize}
  \item The simplest solution is to use Quilt
    \begin{itemize}
    \item Quilt is a tool that allows to maintain a stack of patches
      over source code
    \item Makes it easy to add, remove modifications from a patch, to
      add and remove patches from stack and to update them
    \item The stack of patches can be integrated into your version
      control system
    \item \url{https://savannah.nongnu.org/projects/quilt/}
    \end{itemize}
  \item Another solution is to use a version control system
    \begin{itemize}
    \item Import the original component version into your version
      control system
    \item Maintain your changes in a separate branch
    \end{itemize}
  \end{itemize}
\end{frame}

\subsection[Networking tools]{Tools for the target device: Networking}

\begin{frame}
  \frametitle{ssh server and client: Dropbear}
  \url{http://matt.ucc.asn.au/dropbear/dropbear.html}
  \begin{itemize}
  \item Very small memory footprint ssh server for embedded systems
  \item Satisfies most needs. Both client and server!
  \item Size: 110 KB, statically compiled with uClibc on i386.\\
    (OpenSSH client and server: approx 1200 KB, dynamically compiled
    with glibc on i386)
  \item Useful to:
    \begin{itemize}
    \item Get a remote console on the target device
    \item Copy files to and from the target device (scp or rsync -e
      ssh).
    \end{itemize}
  \item An alternative to OpenSSH, used on desktop and server systems.
  \end{itemize}
\end{frame}

\begin{frame}
  \frametitle{Benefits of a web server interface}

  Many network enabled devices can just have a network interface

  \begin{itemize}
  \item Examples: modems / routers, IP cameras, printers...
  \item No need to develop drivers and applications for computers
    connected to the device. No need to support multiple operating
    systems!
  \item Just need to develop static or dynamic HTML pages (possibly
    with powerful client-side JavaScript).\\
    Easy way of providing access to device information and parameters.
  \item Reduced hardware costs (no LCD, very little storage space
    needed)
  \end{itemize}
\end{frame}

\begin{frame}
  \frametitle{Web servers}
  \begin{columns}
    \column{0.8\textwidth}
    \begin{itemize}
    \item {\em BusyBox http server}: \url{http://busybox.net}
      \begin{itemize}
      \item Tiny: only adds 9 K to BusyBox (dynamically linked with
        glibc on i386, with all features enabled.)
      \item Sufficient features for many devices with a web interface,
        including CGI, http authentication and script support (like
        PHP, with a separate interpreter).
      \item License: GPL
      \end{itemize}
    \item {\em lighttpd}: \url{http://lighttpd.net}\\
      Low footprint server good at managing high loads.\\
      May be useful in embedded systems too
    \item Other possibilities: {\em Boa}, {\em thttpd}, {\em nginx},
      etc
    \end{itemize}
    \column{0.2\textwidth}
    \includegraphics[width=0.9\textwidth]{slides/sysdev-embedded-linux/busybox.png}\\
    \includegraphics[width=0.9\textwidth]{slides/sysdev-embedded-linux/lighttpd.png}\\
  \end{columns}
\end{frame}

\begin{frame}
  \frametitle{Network utilities (1)}
  \begin{itemize}
  \item {\bf avahi} is an implementation of Multicast DNS Service
    Discovery, that allows programs to publish and discover services
    on a local network
  \item {\bf bind}, a DNS server
  \item {\bf iptables}, the user space tools associated to the Linux firewall, Netfilter
  \item {\bf iw and wireless tools}, the user space tools associated to Wireless devices
  \item {\bf netsnmp}, implementation of the SNMP protocol
  \item {\bf openntpd}, implementation of the Network Time Protocol,
    for clock synchronization
  \item {\bf openssl}, a toolkit for SSL and TLS connections
  \end{itemize}
\end{frame}

\begin{frame}
  \frametitle{Network utilities (2)}
  \begin{itemize}
  \item {\bf pppd}, implementation of the Point to Point Protocol,
    used for dial-up connections
  \item {\bf samba}, implements the SMB and CIFS protocols, used by
    Windows to share files and printers
  \item {\bf coherence}, a UPnP/DLNA implementation
  \item {\bf vsftpd}, proftpd, FTP servers
  \end{itemize}
\end{frame}

\subsection[System utilities]{Tools for the target device: System
  utilities}

\begin{frame}
  \frametitle{System utilities}
  \begin{itemize}
  \item {\bf dbus}, an inter-application object-oriented communication bus
  \item {\bf gpsd}, a daemon to interpret and share GPS data
  \item {\bf libraw1394}, raw access to Firewire devices
  \item {\bf libusb}, a user space library for accessing USB devices
    without writing an in-kernel driver
  \item Utilities for kernel subsystems: {\bf i2c-tools} for I2C, {\bf
      input-tools} for input, {\bf mtd-utils} for MTD devices, {\bf
      usbutils} for USB devices
  \end{itemize}
\end{frame}

\subsection[Language Interpreters]{Tools for the target device:
  Language interpreters}

\begin{frame}
  \frametitle{Language interpreters}
  \begin{itemize}
  \item Interpreters for the most common scripting languages are
    available. Useful for
    \begin{itemize}
    \item Application development
    \item Web services development
    \item Scripting
    \end{itemize}
  \item Languages supported
    \begin{itemize}
    \item Lua
    \item Python
    \item Perl
    \item Ruby
    \item TCL
    \item PHP
    \end{itemize}
  \end{itemize}
\end{frame}

\subsection[Multimedia tools]{Tools for the target device: Audio,
  video and multimedia}

\begin{frame}
  \frametitle{Audio, video and multimedia}
  \begin{itemize}
  \item {\bf GStreamer}, a multimedia framework
    \begin{itemize}
    \item Allows to decode/encode a wide variety of codecs.
    \item Supports hardware encoders and decoders through plugins,
      proprietary/specific plugins are often provided by SoC vendors.
    \end{itemize}
  \item {\bf alsa-lib}, the user space tools associated to the ALSA sound
    kernel subsystem
  \item Directly using encoding and decoding libraries, if you decide
    not to use GStreamer:\\
    libavcodec, libogg, libtheora, libvpx, flac, libvorbis, libmad,
    libsndfile, speex, etc.
  \end{itemize}
\end{frame}

\subsection[Graphical toolkits]{Tools for the target device: Graphical
  toolkits}

\subsection[Low-level toolkits]{Graphical toolkits:
``Low-level'' solutions and layers}

\begin{frame}
  \frametitle{DirectFB (1)}
  \begin{itemize}
  \item Low-level graphical library
    \begin{itemize}
    \item Lines, rectangles, triangles drawing and filling
    \item Blitting, flipping
    \item Text drawing
    \item Windows and transparency
    \item Image loading and video display
    \end{itemize}
  \item But also handles input event handling: mouse, keyboard, joystick,
    touchscreen, etc.
  \item Provides accelerated graphic operations on a few hardware
    platforms
  \item Single-application by default, but multiple applications can
    share the framebuffer thanks to {\em fusion}
  \item License: LGPL 2.1
  \item \url{http://www.directfb.org}
  \end{itemize}
\end{frame}

\begin{frame}
  \frametitle{DirectFB: architecture}
  \begin{center}
    \includegraphics[height=0.8\textheight]{slides/sysdev-embedded-linux/directfb-architecture.pdf}
  \end{center}
\end{frame}

\begin{frame}
  \frametitle{DirectFB: usage (1)}
  \begin{itemize}
  \item Multimedia applications
    \begin{itemize}
    \item For example the Disko framework, for set-top box related
      applications
    \end{itemize}
  \item ``Simple'' graphical applications
    \begin{itemize}
    \item Industrial control
    \item Device control with limited number of widgets
    \end{itemize}
  \item Visualization applications
  \item As a lower layer for higher-level graphical libraries
  \end{itemize}
\end{frame}

\begin{frame}
  \frametitle{DirectFB: usage (2)}
  \frametitle{DirectFB: architecture}
  \begin{center}
    \includegraphics[height=0.8\textheight]{slides/sysdev-embedded-linux/directfb-morphine.png}
  \end{center}
\end{frame}

\begin{frame}
  \frametitle{X.org - KDrive}
  \begin{columns}[T]
    \column{0.8\textwidth}
    \begin{itemize}
    \item Stand-alone simplified version of the X server, for embedded
      systems
      \begin{itemize}
      \item Formerly know as Tiny-X
      \item Kdrive is integrated in the official X.org server
      \end{itemize}
    \item Works on top of the Linux frame buffer, thanks to the Xfbdev
      variant of the server
    \item Real X server
      \begin{itemize}
      \item Fully supports the X11 protocol: drawing, input event
        handling, etc.
      \item Allows to use any existing X11 application or library
      \end{itemize}
    \item Actively developed and maintained.
    \item X11 license
    \item \url{http://www.x.org}
    \end{itemize}
    \column{0.2\textwidth}
    \includegraphics[width=\textwidth]{slides/sysdev-embedded-linux/xorg.png}
  \end{columns}
\end{frame}

\begin{frame}
  \frametitle{Kdrive: architecture}
  \begin{center}
    \includegraphics[height=0.8\textheight]{slides/sysdev-embedded-linux/xorg-architecture.pdf}
  \end{center}
\end{frame}

\begin{frame}
  \frametitle{Kdrive: usage}
  \begin{itemize}
  \item Can be directly programmed using Xlib / XCB
    \begin{itemize}
    \item Low-level graphic library
    \item Probably doesn't make sense since DirectFB is a more
      lightweight solution for an API of roughly the same level (no
      widgets)
    \end{itemize}
  \item Or, usually used with a toolkit on top of it
    \begin{itemize}
    \item Gtk
    \item Qt
    \item Enlightment Foundation Libraries
    \item Others: Fltk, WxEmbedded, etc
    \end{itemize}
  \end{itemize}
\end{frame}

\begin{frame}
  \frametitle{Wayland}
  \begin{columns}[T]
    \column{0.8\textwidth}
    \begin{itemize}
    \item Intended to be a simpler replacement for X
    \item {\em Wayland is a protocol for a compositor to talk to
    its clients as well as a C library implementation of that protocol.}
    \item Weston: a minimal and fast reference implementation
          of a Wayland compositor, and is suitable for many embedded
          and mobile use cases.
    \item Still experimental. Porting Gtk and Qt
          to Wayland is not complete yet, for example.
    \item \url{http://wayland.freedesktop.org/}
    \end{itemize}
    \column{0.2\textwidth}
    \includegraphics[width=\textwidth]{slides/sysdev-embedded-linux/wayland.png}
  \end{columns}
\end{frame}

\begin{frame}
  \frametitle{Wayland: architecture}
  \begin{center}
    \includegraphics[height=0.8\textheight]{slides/sysdev-embedded-linux/wayland-architecture.pdf}
  \end{center}
\end{frame}


\subsection[High-level Toolkits]{Graphical toolkits: ``High-level''
  solutions}

\begin{frame}
  \frametitle{Gtk}
  \begin{columns}
    \column{0.7\textwidth}
    \begin{itemize}
    \item The famous toolkit, providing widget-based high-level APIs to
      develop graphical applications
    \item Standard API in C, but bindings exist for various languages:
      C++, Python, etc.
    \item Works on top of X.org.
    \item No windowing system, a lightweight window manager needed to
      run several applications. Possible solution: Matchbox.
    \item License: LGPL
    \item \url{http://www.gtk.org}
    \end{itemize}
    \column{0.3\textwidth}
    \includegraphics[width=0.8\textwidth]{slides/sysdev-embedded-linux/gtk-backends.pdf}
  \end{columns}
\end{frame}

\begin{frame}
  \frametitle{Gtk stack components}
  \begin{itemize}
  \item {\bf Glib}, core infrastructure
    \begin{itemize}
    \item Object-oriented infrastructure GObject
    \item Event loop, threads, asynchronous queues, plug-ins, memory
      allocation, I/O channels, string utilities, timers, date and
      time, internationalization, simple XML parser, regular
      expressions
    \item Data types: memory slices and chunks, linked lists, arrays,
      trees, hash tables, etc.
    \end{itemize}
  \item {\bf Pango}, internationalization of text handling
  \item {\bf ATK}, accessibility toolkit
  \item {\bf Cairo}, vector graphics library
  \item {\bf Gtk+}, the widget library itself
  \item {\em The Gtk stack is a complete framework to develop applications}
  \end{itemize}
\end{frame}

\begin{frame}
  \frametitle{Gtk examples (1)}
  \begin{columns}
    \column{0.5\textwidth}
    \includegraphics[width=\textwidth]{slides/sysdev-embedded-linux/openmoko-gui.png}\\
    \column{0.5\textwidth}
    Openmoko phone interface
  \end{columns}
\end{frame}

\begin{frame}
  \frametitle{Gtk examples (2)}
    \includegraphics[height=0.6\textheight]{slides/sysdev-embedded-linux/maemo-gui.png}\\
    Maemo tablet / phone interface\\
    GTK is losing traction, however:
    Mer, the descendent of Maemo, is now implemented in EFL (see next
    slides).\\
\end{frame}

\begin{frame}
  \frametitle{Qt (1)}
  \begin{itemize}
  \item The other famous toolkit, providing widget-based high-level APIs to
    develop graphical applications
  \item Implemented in C++
    \begin{itemize}
    \item the C++ library is required on the target system
    \item standard API in C++, but with bindings for other languages
    \end{itemize}
  \item Works either on top of
    \begin{itemize}
    \item Framebuffer
    \item X11
    \item DirectFB back-end integrated in version 4.4, which allows to take
      advantage of the acceleration provided by DirectFB drivers
    \end{itemize}
  \end{itemize}
\end{frame}

\begin{frame}
  \frametitle{Qt (2)}
  \begin{itemize}
  \item Qt is more than just a graphical toolkit, it also offers a
    complete development framework: data structures, threads, network,
    databases, XML, etc.
  \item See our presentation {\em Qt for non graphical applications}
    presentation at ELCE 2011 (Thomas Petazzoni):
    \url{http://j.mp/W4PK85}
  \item Qt Embedded has an integrated windowing system, allowing
    several applications to share the same screen
  \item Very well documented
  \item Since version 4.5, available under the LGPL, allowing
    proprietary applications
  \end{itemize}
\end{frame}

\begin{frame}
  \frametitle{Qt's usage}
  \begin{columns}
    \column{0.5\textwidth}
    \includegraphics[width=\textwidth]{slides/sysdev-embedded-linux/dash-express-gui.jpg}\\
    Qt on the Dash Express navigation system
    \column{0.5\textwidth}
    \includegraphics[width=\textwidth]{slides/sysdev-embedded-linux/roku-netflix-gui.jpg}\\
    Qt on the Netflix player by Roku
  \end{columns}
\end{frame}

\begin{frame}
  \frametitle{Other less frequent solutions}
  \begin{itemize}
  \item Enlightenment Foundation Libraries (EFL)
    \begin{itemize}
    \item Very powerful. Supported by Samsung, Intel and Free.fr.
    \item Work on top of X or Wayland.
    \item \url{http://www.enlightenment.org/p.php?p=about/efl}
    \end{itemize}
  \end{itemize}
\end{frame}

\subsection[Databases]{Tools for the target device: Databases}

\begin{frame}
  \frametitle{Lightweight database - SQLite}
  \url{http://www.sqlite.org}

  \begin{itemize}
  \item SQLite is a small C library that implements a self-contained,
    embeddable, lightweight, zero-configuration SQL database engine
  \item The database engine of choice for embedded Linux systems
    \begin{itemize}
    \item Can be used as a normal library
    \item Can be directly embedded into a application, even a
      proprietary one since SQLite is released in the public domain
    \end{itemize}
  \end{itemize}
\end{frame}

\subsection[Web Browsers]{Tools for the target device: Web browsers}

\begin{frame}
  \frametitle{WebKit}
  \begin{columns}[T]
    \column{0.8\textwidth}
    \url{http://webkit.org/}
    \begin{itemize}
    \item Web browser engine. Application framework that can be used
      to develop web browsers.
    \item License: portions in LGPL and others in BSD. Proprietary
      applications allowed.
    \item Used by many web browsers: Safari, iPhone et Android default
      browsers ... Google Chrome now uses a fork of its WebCore component).
      Used by e-mail clients too to render HTML:
      \url{http://trac.webkit.org/wiki/Applications\%20using\%20WebKit}
    \item Multiple graphical back-ends: Qt4, GTK, EFL...
    \item You could use it to create your custom browser.
    \end{itemize}
    \column{0.2\textwidth}
    \includegraphics[width=\textwidth]{slides/sysdev-embedded-linux/webkit.png}
  \end{columns}
\end{frame}

\subsection{Example of components used in real devices}

\begin{frame}
  \frametitle{Industrial applications}
  \begin{itemize}
  \item In many industrial applications, the system is only
    responsible for monitoring and control a device
  \item Such a system is usually relatively simple in terms of
    components
    \begin{itemize}
    \item Kernel
    \item BusyBox
    \item C library
    \item Applications relying directly on the C library, sometimes
      using the real-time capabilities of the Linux kernel
    \item Sometimes a Web server for remote control, or another server
      implementing a custom protocol
    \end{itemize}
  \end{itemize}
\end{frame}

\begin{frame}
  \frametitle{Digital Photo Frame: requirements}
  \begin{itemize}
  \item Example taken from a conference of Matt Porter, Embedded Alley
    at ELC 2008
  \item Hardware: ARM SoC with DSP, audio, 800x600 LCD, MMC/SD, NAND,
    buttons, speakers
  \item The photo frame must be able to
    \begin{itemize}
    \item Display to the LCD
    \item Detect SD card insertion, notify applications of the
      insertion so that applications can build a catalog of the
      pictures on the SD card
    \item Modern 3D GUI with nice transitions
    \item Navigation through buttons
    \item Support audio playback (MP3, playlists, ID3 tag)
    \item JPEG resizing and rotation
    \end{itemize}
  \end{itemize}
\end{frame}

\begin{frame}
  \frametitle{Digital Photo Frame: components (1)}
  \begin{itemize}
  \item Base system
    \begin{itemize}
    \item Components present in virtually all embedded Linux systems
    \item The U-Boot bootloader
    \item Linux Kernel
      \begin{itemize}
      \item Drivers for SD/MMC, framebuffer, sound, input devices
      \end{itemize}
    \item Busybox
    \item Build system, in this case was OpenEmbedded
    \item Components: {\bf u-boot}, {\bf Linux}, {\bf busybox}
    \end{itemize}
  \end{itemize}
\end{frame}

\begin{frame}
  \frametitle{Digital Photo Frame: components (2)}
  \begin{itemize}
  \item Event handling to detect SD card insertion
    \begin{itemize}
    \item \code{udev}, that receives events from the kernel, creates device
      nodes, and sends events to \code{hal}
    \item \code{hal}, which maintains a database of available devices and
      provides a \code{dbus} API
    \item \code{dbus} to connect \code{hal} with the application. The application
      subscribes to \code{hal} event through \code{dbus} and gets notified when
      they are triggered
    \item Components: {\bf udev}, {\bf hal}, {\bf dbus}
    \end{itemize}
  \end{itemize}
\end{frame}

\begin{frame}
  \frametitle{Digital Photo Frame: components (3)}
  \begin{itemize}
  \item JPEG display
    \begin{itemize}
    \item {\bf libjpeg} to decode the pictures
    \item {\bf jpegtran} to resize and rotate them
    \item {\bf FIM} (Fbi Improved) for dithering
    \end{itemize}
  \item MP3 support
    \begin{itemize}
    \item {\bf libmad} for playing
    \item {\bf libid3} for ID3 tags reading
    \item {\bf libm3u} to support playlists
    \item Used vendor-provided components to leverage the DSP to play
      MP3
    \end{itemize}
  \end{itemize}
\end{frame}

\begin{frame}
  \frametitle{Digital Photo Frame: components (4)}
  \begin{itemize}
  \item 3D interface
    \begin{itemize}
    \item Vincent, an open-source implementation of OpenGL ES
    \item Clutter, higher-level API to develop 3D applications
    \end{itemize}
  \item Application itself
    \begin{itemize}
    \item Manages media events
    \item Uses the JPEG libraries to decode and render pictures
    \item Receives Linux input events from buttons and draws
      OpenGL-based UI developed with Clutter
    \item Manage a user-defined configuration
    \item Play the music with the MP3-related libraries
    \item Display photo slideshow
    \end{itemize}
  \end{itemize}
\end{frame}

\subsection{Commercial embedded Linux solutions}

\begin{frame}
  \frametitle{Commercial embedded Linux solutions}
  Caution: {\em commercial} doesn't mean {\em proprietary!}
  \begin{itemize}
  \item Vendors play fair with the GPL and do make their source code
    available to their users, and most of the time, eventually to the
    community.
    \begin{itemize}
    \item As long as they distribute the sources to their users, the
      GPL doesn't require vendors to share their sources with any
      third party.
    \end{itemize}
  \item Graphical toolkits developed by the vendors are usually
    proprietary, trying to make it easier to create and embedded Linux
    systems.
  \item Major players: Wind River, Montavista, TimeSys
  \end{itemize}
\end{frame}

\begin{frame}
  \frametitle{Commercial solution strengths}
  \begin{columns}
    \column{0.5\textwidth}
    \begin{itemize}
    \item Technical advantages
      \begin{itemize}
      \item Well tested and supported kernel and tool versions
      \item Often supporting patches not supported by the
        mainline kernel yet (example: real-time patches)
      \end{itemize}
    \item Complete development tool sets: kernels, toolchains,
      utilities, binaries for impressive lists of target platforms
    \item Integrated utilities for automatic kernel image, initramfs
      and filesystem generation.
    \end{itemize}
    \column{0.5\textwidth}
    \begin{itemize}
    \item Graphical developments tools
    \item Development tools available on multiple platforms: GNU /
      Linux, Solaris, Windows...
    \item Support services
      \begin{itemize}
      \item Useful if you don't have your own support resources
      \item Long term support commitment, even for versions considered
        as obsolete by the community, but not by your users!
      \end{itemize}
    \end{itemize}
  \end{columns}
\end{frame}

\begin{frame}
  \frametitle{Commercial or community solutions?}
  {\bf Commercial distributions and tool sets}
  \begin{itemize}
  \item Best if you don't have your own support resources and have a
    sufficient budget
  \item Help focusing on your primary job: making an embedded device.
  \item You can even subcontract driver development to the vendor
  \end{itemize}
  {\bf Community distributions and tools}
  \begin{itemize}
  \item Best if you are on a tight budget
  \item Best if you are willing to build your own embedded Linux
    expertise, investigate issues by yourselves, and train your own
    support resources.
  \end{itemize}
  In any case, your products are based on Free Software!
\end{frame}
