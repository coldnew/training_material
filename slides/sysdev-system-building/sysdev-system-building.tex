\section{Embedded Linux system development - part 1}

\begin{frame}
  \frametitle{Contents}
  \begin{itemize}
  \item \code{Using open-source components}
  \item \code{Tools for the target device}
    \begin{itemize}
    \item \code{Networking}
    \item \code{System utilities}
    \item \code{Language interpreters}
    \item \code{Audio, video and multimedia}
    \item \code{Graphical toolkits}
    \item \code{Databases}
    \item \code{Web browsers}
    \end{itemize}
  \item {\bf System building}
  \item \code{Commercial tool sets and distributions}
  \end{itemize}
\end{frame}

\subsection{System building}

\begin{frame}
  \frametitle{System building: goal and solutions}
  \begin{itemize}
  \item Goal
    \begin{itemize}
    \item Integrate all the software components, both third-party and
      in-house, into a working root filesystem
    \item It involves the download, extraction, configuration,
      compilation and installation of all components, and possibly
      fixing issues and adapting configuration files
    \end{itemize}
  \item Several solutions
    \begin{itemize}
    \item Manually
    \item System building tools
    \item Distributions or ready-made filesystems
    \end{itemize}
  \end{itemize}
\end{frame}

\begin{frame}
  \frametitle{System building: manually}
  \begin{itemize}
  \item Manually building a target system involves downloading,
    configuring, compiling and installing all the components of the
    system.
  \item All the libraries and dependencies must be configured,
    compiled and installed in the right order.
  \item Sometimes, the build system used by libraries or applications
    is not very cross-compile friendly, so some adaptations are
    necessary.
  \item There is no infrastructure to reproduce the build from
    scratch, which might cause problems if one component needs to be
    changed, if somebody else takes over the project, etc.
  \end{itemize}
\end{frame}

\begin{frame}
  \frametitle{System building: manually (2)}
  \begin{itemize}
  \item Manual system building is not recommended for production
    projects
  \item However, using automated tools often requires the developer to
    dig into specific issues
  \item Having a basic understanding of how a system can be built
    manually is therefore very useful to fix issues encountered with
    automated tools
    \begin{itemize}
    \item \st{We will first study manual system building, and during a
      practical lab, create a system using this method}
    \item Then, we will study the automated tools available, and use
      one of them during a lab
    \end{itemize}
  \end{itemize}
\end{frame}

\begin{frame}
  \frametitle{System foundations}
  \begin{itemize}
  \item A basic root file system needs at least
    \begin{itemize}
    \item A traditional directory hierarchy, with \code{/bin},
      \code{/etc}, \code{/lib}, \code{/root}, \code{/usr/bin},
      \code{/usr/lib}, \code{/usr/share}, \code{/usr/sbin},
      \code{/var}, \code{/sbin}
    \item A set of basic utilities, providing at least the \code{init}
      program, a shell and other traditional Unix command line
      tools. This is usually provided by {\bf Busybox}
    \item The C library and the related libraries (thread, math, etc.)
      installed in \code{/lib}
    \item A few configuration files, such as \code{/etc/inittab}, and
      initialization scripts in \code{/etc/init.d}
    \end{itemize}
  \item On top of this foundation common to most embedded Linux
    system, we can add third-party or in-house components
  \end{itemize}
\end{frame}

\begin{frame}
  \frametitle{Target and build spaces}
  \begin{itemize}
  \item The system foundation, Busybox and C library, are the core of
    the target root filesystem
  \item However, when building other components, one must distinguish
    two directories
    \begin{itemize}
    \item The {\em target} space, which contains the target root
      filesystem, everything that is needed for {\bf execution} of the
      application
    \item The {\em build} space, which will contain a lot more files
      than the {\em target} space, since it is used to keep everything
      needed to {\bf compile} libraries and applications. So we must
      keep the headers, documentation, and other configuration files
    \end{itemize}
  \end{itemize}
\end{frame}

\begin{frame}
  \frametitle{Build systems}
  \begin{itemize}
  \item Each open-source component comes with a mechanism to
    configure, compile and install it
    \begin{itemize}
    \item A basic \code{Makefile}
      \begin{itemize}
      \item Need to read the \code{Makefile} to understand how it
        works and how to tweak it for cross-compilation
      \end{itemize}
    \item A build system based on the {\em Autotools}
      \begin{itemize}
      \item As this is the most common build system, we will study it
        in details
      \end{itemize}
    \item CMake, \url{http://www.cmake.org/}
      \begin{itemize}
      \item Newer and simpler than the {\em autotools}. Used by large
        projects such as KDE or Second Life
      \end{itemize}
    \item Scons, \url{http://www.scons.org/}
    \item Waf, \url{http://code.google.com/p/waf/}
    \item Other manual build systems
    \end{itemize}
  \end{itemize}
\end{frame}

\begin{frame}
  \frametitle{Autotools and friends}
  \begin{itemize}
  \item A family of tools, which associated together form a complete
    and extensible build system
    \begin{itemize}
    \item {\bf autoconf} is used to handle the configuration of the
      software package
    \item {\bf automake} is used to generate the Makefiles needed to
      build the software package
    \item {\bf pkgconfig} is used to ease compilation against already
      installed shared libraries
    \item {\bf libtool} is used to handle the generation of shared
      libraries in a system-independent way
    \end{itemize}
  \item Most of these tools are old and relatively complicated to use,
    but they are used by a majority of free software packages
    today. One must have a basic understanding of what they do and how
    they work.
  \end{itemize}
\end{frame}

\begin{frame}
  \frametitle{automake / autoconf / autoheader}
  \begin{center}
    \includegraphics[width=0.9\textwidth]{slides/sysdev-embedded-linux/autotools.pdf}
  \end{center}
\end{frame}

\begin{frame}
  \frametitle{automake / autoconf}
  \begin{itemize}
  \item Files written by the developer
    \begin{itemize}
    \item \code{configure.in} describes the configuration options and
      the checks done at configure time
    \item \code{Makefile.am} describes how the software should be
      built
    \end{itemize}
  \item The \code{configure} script and the \code{Makefile.in} files
    are generated by \code{autoconf} and \code{automake} respectively.
    \begin{itemize}
    \item They should never be modified directly
    \item They are usually shipped pre-generated in the software
      package, because there are several versions of \code{autoconf}
      and \code{automake}, and they are not completely compatible
    \end{itemize}
  \item The \code{Makefile} files are generated at configure time, before
    compiling
    \begin{itemize}
    \item They are never shipped in the software package.
    \end{itemize}
  \end{itemize}
\end{frame}

\begin{frame}
  \frametitle{Configuring and compiling: native case}
  \begin{itemize}
  \item The traditional steps to configure and compile an autotools
    based package are
    \begin{itemize}
    \item Configuration of the package\\
      \code{./configure}
    \item Compilation of the package\\
      \code{make}
    \item Installation of the package\\
      \code{make install}
    \end{itemize}
  \item Additional arguments can be passed to the \code{./configure}
    script to adjust the component configuration.
  \item Only the \code{make install} needs to be done as root if the
    installation should take place system-wide
  \end{itemize}
\end{frame}

\begin{frame}
  \frametitle{Configuring and compiling: cross case (1)}
  \begin{itemize}
  \item For cross-compilation, things are a little bit more complicated.
  \item At least some of the environment variables \code{AR},
    \code{AS}, \code{LD}, \code{NM}, \code{CC}, \code{GCC},
    \code{CPP}, \code{CXX}, \code{STRIP}, \code{OBJCOPY} must be
    defined to point to the proper cross-compilation tools. The host
    tuple is also by default used as prefix.
  \item \code{configure} script arguments:
    \begin{itemize}
    \item \code{--host}: mandatory but a bit confusing.
     Corresponds to the {\em target} platform the code will run on.
     Example: \code{--host=arm-linux}
    \item \code{--build}: build system. Automatically detected.
    \item \code{--target} is only for tools generating code.
    \end{itemize}
  \item It is recommended to pass the \code{--prefix} argument. It
    defines from which location the software will run in the target
    environment. Usually, \code{/usr} is fine.
  \end{itemize}
\end{frame}

\begin{frame}[fragile]
  \frametitle{Configuring and compiling: cross case (2)}
  \begin{itemize}
  \item If one simply runs \code{make install}, the software will be
    installed in the directory passed as \code{--prefix}. For
    cross-compiling, one must pass the \code{DESTDIR} argument to
    specify where the software must be installed.
  \item Making the distinction between the prefix (as passed with
    \code{--prefix} at configure time) and the destination directory (as
    passed with \code{DESTDIR} at installation time) is very important.
  \item Example:
\small
\begin{block}{}
\begin{verbatim}
export PATH=/usr/local/arm-linux/bin:$PATH
export CC=arm-linux-gcc
export STRIP=arm-linux-strip
./configure --host=arm-linux --prefix=/usr
make
make DESTDIR=$HOME/work/rootfs install
\end{verbatim}
\end{block}
  \end{itemize}
\end{frame}

\begin{frame}
  \frametitle{Installation (1)}
  \begin{itemize}
  \item The autotools based software packages provide both a
    \code{install} and \code{install-strip} make targets, used to
    install the software, either stripped or unstripped.
  \item For applications, the software is usually installed in
    \code{<prefix>/bin}, with configuration files in
    \code{<prefix>/etc} and data in
    \code{<prefix>/share/<application>/}
  \item The case of libraries is a little more complicated:
    \begin{itemize}
    \item In \code{<prefix>/lib}, the library itself (a
      \code{.so.<version>}), a few symbolic links, and
      the libtool description file (a \code{.la} file)
    \item The {\em pkgconfig} description file in
      \code{<prefix>/lib/pkgconfig}
    \item Include files in \code{<prefix>/include/}
    \item Sometimes a \code{<libname>-config} program in
      \code{<prefix>/bin}
    \item Documentation in \code{<prefix>/share/man} or
      \code{<prefix>/share/doc/}
    \end{itemize}
  \end{itemize}
\end{frame}

\begin{frame}
\frametitle{Installation (2)}

Contents of \code{usr/lib} after installation of {\em libpng} and {\em
  zlib}

\scriptsize
\begin{itemize}
\item {\em libpng} libtool description files\\
  \code{./lib/libpng12.la}\\
  \code{./lib/libpng.la -> libpng12.la}
\item {\em libpng} static version\\
  \code{./lib/libpng12.a}\\
  \code{./lib/libpng.a -> libpng12.a}
\item {\em libpng} dynamic version\\
  \code{./lib/libpng.so.3.32.0}\\
  \code{./lib/libpng12.so.0.32.0}\\
  \code{./lib/libpng12.so.0 -> libpng12.so.0.32.0}\\
  \code{./lib/libpng12.so -> libpng12.so.0.32.0}\\
  \code{./lib/libpng.so -> libpng12.so}\\
  \code{./lib/libpng.so.3 -> libpng.so.3.32.0}
\item {\em libpng} pkg-config description files\\
  \code{./lib/pkgconfig/libpng12.pc}\\
  \code{./lib/pkgconfig/libpng.pc -> libpng12.pc}
\item {\em zlib} dynamic version\\
  \code{./lib/libz.so.1.2.3}\\
  \code{./lib/libz.so -> libz.so.1.2.3}\\
  \code{./lib/libz.so.1 -> libz.so.1.2.3}
\end{itemize}
\end{frame}

\begin{frame}
  \frametitle{Installation in the build and target spaces}
  \begin{itemize}
  \item From all these files, everything except documentation is
    necessary to build an application that relies on libpng.
    \begin{itemize}
    \item These files will go into the {\em build space}
    \end{itemize}
  \item However, only the library \code{.so} binaries in
    \code{<prefix>/lib} and some symbolic links are needed to execute
    the application on the target.
    \begin{itemize}
    \item Only these files will go in the {\em target space}
    \end{itemize}
  \item The build space must be kept in order to build other
    applications or recompile existing applications.
  \end{itemize}
\end{frame}

\begin{frame}
  \frametitle{pkg-config}
  \begin{itemize}
  \item \code{pkg-config} is a tool that allows to query a small
    database to get information on how to compile programs that depend
    on libraries
  \item The database is made of \code{.pc} files, installed by default in
    \code{<prefix>/lib/pkgconfig/}.
  \item \code{pkg-config} is used by the \code{configure} script to get the
    library configurations
  \item It can also be used manually to compile an application:\\
    \code{arm-linux-gcc -o test test.c $(pkg-config --libs --cflags thelib)}
  \item By default, \code{pkg-config} looks in \code{/usr/lib/pkgconfig} for
    the \code{*.pc} files, and assumes that the paths in these files
    are correct.
  \item \code{PKG_CONFIG_PATH} allows to set another location for the
    \code{*.pc} files and \code{PKG_CONFIG_SYSROOT_DIR} to prepend a
    prefix to the paths mentioned in the \code{.pc} files.
  \end{itemize}
\end{frame}

\begin{frame}
  \frametitle{Let's find the libraries}
  \begin{itemize}
  \item When compiling an application or a library that relies on
    other libraries, the build process by default looks in
    \code{/usr/lib} for libraries and \code{/usr/include} for headers.
  \item The first thing to do is to set the \code{CFLAGS} and
    \code{LDFLAGS} environment
    variables:\\
    \code{export CFLAGS=-I/my/build/space/usr/include/}\\
    \code{export LDFLAGS=-L/my/build/space/usr/lib}
  \item The libtool files (\code{.la} files) must be modified because they
    include the absolute paths of the libraries:\\
    \code{- libdir='/usr/lib'}\\
    \code{+ libdir='/my/build/space/usr/lib}'
  \item The \code{PKG_CONFIG_PATH} environment variable must be set to
    the location of the .pc files and the
    \code{PKG_CONFIG_SYSROOT_DIR} variable must be set to the build
    space directory.
\end{itemize}
\end{frame}

\subsection{System building tools}

\begin{frame}
  \frametitle{System building tools: principle}
  \begin{itemize}
  \item Different tools are available to automate the process of
    building a target system, including the kernel, and sometimes the
    toolchain.
  \item They automatically download, configure, compile and install
    all the components in the right order, sometimes after applying
    patches to fix cross-compiling issues.
  \item They already contain a large number of packages, that should
    fit your main requirements, and are easily extensible.
  \item The build becomes reproducible, which allows to easily change
    the configuration of some components, upgrade them, fix bugs, etc.
  \end{itemize}
\end{frame}

\begin{frame}
  \frametitle{Available system building tools} Large choice of tools
  \small
  \begin{itemize}
  \item {\bf Buildroot}, developed by the community\\
    \url{http://www.buildroot.net}
  \item {\bf PTXdist}, developed by Pengutronix\\
    \url{http://pengutronix.de/software/ptxdist/}
  \item {\bf OpenWRT}, originally a fork of Buildroot for wireless routers,
    now a more generic project\\
    \url{http://www.openwrt.org}
  \item {\bf LTIB}. Good support for Freescale boards, but small community\\
    \url{http://ltib.org/}
  \item {\bf OpenEmbedded}, more flexible but also far more complicated\\
    \url{http://www.openembedded.org}, its industrialized version {\bf
      Yocto} and vendor-specific derivatives such as {\bf Arago}
  \item Vendor specific tools (silicon vendor or embedded Linux
    vendor)
  \end{itemize}
\end{frame}

\begin{frame}
  \frametitle{Buildroot (1)}
  \begin{itemize}
  \item Allows to build a toolchain, a root filesystem image with many
    applications and libraries, a bootloader and a kernel image
    \begin{itemize}
    \item Or any combination of the previous items
    \end{itemize}
  \item Supports building uClibc, glibc, eglibc and musl toolchains,
    either built by Buildroot, or external
  \item Over 1200+ applications or libraries integrated, from basic
    utilities to more elaborate software stacks: X.org, GStreamer, Qt,
    Gtk, WebKit, Python, PHP, etc.
  \item Good for small to medium embedded systems, with a fixed set of
    features
    \begin{itemize}
    \item No support for generating packages (\code{.deb} or
      \code{.ipk})
    \item Needs complete rebuild for most configuration changes.
    \end{itemize}
  \item Active community, releases published every 3 months.
  \end{itemize}
\end{frame}

\begin{frame}
  \frametitle{Buildroot (2)}
  \begin{columns}
    \column{0.6\textwidth}
    \begin{itemize}
    \item Configuration takes place through a \code{*config} interface similar to the
      kernel\\
      \code{make menuconfig}
    \item Allows to define
      \begin{itemize}
      \item Architecture and specific CPU
      \item Toolchain configuration
      \item Set of applications and libraries to integrate
      \item Filesystem images to generate
      \item Kernel and bootloader configuration
      \end{itemize}
    \item Build by just running\\
      \code{make}
    \end{itemize}
    \column{0.4\textwidth}
    \includegraphics[width=\textwidth]{slides/sysdev-embedded-linux/buildroot-screenshot.png}
  \end{columns}
\end{frame}

\begin{frame}
  \frametitle{Buildroot: adding a new package (1)}
  \begin{itemize}
  \item A package allows to integrate a user application or library to
    Buildroot
  \item Each package has its own directory (such as
    \code{package/gqview}). This directory contains:
    \begin{itemize}
    \item A \code{Config.in} file (mandatory), describing the
      configuration options for the package. At least one is needed to
      enable the package. This file must be sourced from
      \code{package/Config.in}
    \item A \code{gqview.mk} file (mandatory), describing how the
      package is built.
    \item Patches (optional). Each file of the form
      \code{gqview-*.patch} will be applied as a patch.
    \end{itemize}
  \end{itemize}
\end{frame}

\begin{frame}[fragile]
  \frametitle{Buildroot: adding a new package (2)}
  \begin{itemize}
  \item For a simple package with a single configuration option to
    enable/disable it, the \code{Config.in} file looks like:
\footnotesize
\begin{block}{}
\begin{verbatim}
config BR2_PACKAGE_GQVIEW
        bool "gqview"
        depends on BR2_PACKAGE_LIBGTK2
        help
          GQview is an image viewer for Unix operating systems

          http://prdownloads.sourceforge.net/gqview
\end{verbatim}
\end{block}
\normalsize
  \item It must be sourced from \code{package/Config.in}:
\small
\begin{block}{}
\begin{verbatim}
source "package/gqview/Config.in"
\end{verbatim}
\end{block}
  \end{itemize}
\end{frame}

\begin{frame}[fragile]
  \frametitle{Buildroot: adding new package (3)}
  \begin{itemize}
  \item Create the \code{gqview.mk} file to describe the build steps
\scriptsize
\begin{block}{}
\begin{verbatim}
GQVIEW_VERSION = 2.1.5
GQVIEW_SOURCE = gqview-$(GQVIEW_VERSION).tar.gz
GQVIEW_SITE = http://prdownloads.sourceforge.net/gqview
GQVIEW_DEPENDENCIES = host-pkgconf libgtk2
GQVIEW_CONF_ENV = LIBS="-lm"

$(eval $(autotools-package))
\end{verbatim}
\end{block}
\normalsize
\item The package directory and the prefix of all variables must be
  identical to the suffix of the main configuration option
  \code{BR2_PACKAGE_GQVIEW}
\item The \code{autotools-package} infrastructure knows how to build
  autotools packages. A more generic \code{generic-package}
  infrastructure is available for packages not using the autotools
  as their build system.
\end{itemize}
\end{frame}

\begin{frame}
  \frametitle{OpenEmbedded / Yocto}
  \begin{itemize}
  \item The most versatile and powerful embedded Linux build system
    \begin{itemize}
    \item A collection of recipes (\code{.bb} files)
    \item A tool that processes the recipes: \code{bitbake}
    \end{itemize}
  \item Integrates 2000+ application and libraries, is highly
    configurable, can generate binary packages to make the system
    customizable, supports multiple versions/variants of the same
    package, no need for full rebuild when the configuration is
    changed.
  \item Configuration takes place by editing various configuration
    files
  \item Good for larger embedded Linux systems, or people looking for
    more configurability and extensibility
  \item Drawbacks: very steep learning curve, very long first build.
  \end{itemize}
\end{frame}

\begin{frame}
  \frametitle{Distributions - Debian}
  \small
  \begin{columns}[T]
    \column{0.8\textwidth}
    Debian GNU/Linux, \url{http://www.debian.org}
    \begin{itemize}
    \item Provides the easiest environment for quickly building prototypes
          and developing applications. Countless runtime and
          development packages available.
    \item But brobably too costly to maintain
          and unnecessarily big for production systems.
    \item Available on ARM, MIPS and PowerPC architectures
    \item Software is compiled natively by default.
    \end{itemize}
    \column{0.2\textwidth}
    \includegraphics[width=\textwidth]{slides/sysdev-embedded-linux/debian.png}\\
  \end{columns}
\end{frame}

\begin{frame}
  \frametitle{Distributions - Others}
  \begin{columns}[T]
    \column{0.7\textwidth}
    Fedora
    \begin{itemize}
    \item \url{http://fedoraproject.org/wiki/Architectures/ARM}
    \item Supported on various recent ARM boards (such as Beaglebone
      Black). Pidora supports Raspberry Pi too.
    \item Supports QEMU emulated ARM boards too (Versatile Express board)
    \item Shipping the same version as for desktops!
    \end{itemize}
    Ubuntu
    \begin{itemize}
    \item Had some releases for ARM mobile multimedia devices, but stopped
          at version 12.04. Now focusing on ARM servers only.
    \end{itemize}
    \column{0.3\textwidth}
    \includegraphics[width=\textwidth]{slides/sysdev-embedded-linux/fedora.png}\\
    \vspace{3.5cm}
    \includegraphics[width=\textwidth]{slides/sysdev-embedded-linux/ubuntu.png}\\
  \end{columns}
\end{frame}

\begin{frame}
  \frametitle{Embedded distributions}
  \small
  \begin{columns}
    \column{0.8\textwidth}
    Distributions designed for specific types of devices
    \begin{itemize}
    \item {\bf Android}: \url{http://www.android.com/}\\
      Google's distribution for phones and tablet PCs.\\
      Except the Linux kernel, very different user space
      than other Linux distributions. Very successful,
      lots of applications available (many proprietary).
    \item {\bf Ångström}: \url{http://www.angstrom-distribution.org/}\\
      Targets PDAs and webpads (Siemens Simpad...)\\
      Binaries available for arm little endian.
    \end{itemize}
    \column{0.2\textwidth}
    \includegraphics[width=\textwidth]{slides/sysdev-embedded-linux/android.png}\\
    \includegraphics[width=\textwidth]{slides/sysdev-embedded-linux/angstrom.png}\\
  \end{columns}
\end{frame}

\begin{frame}
  \frametitle{Application frameworks}
  \small
  \begin{columns}
    \column{0.8\textwidth}
    Not real distributions you can download. Instead, they
    implement middleware running on top of the Linux kernel
    and allowing to develop applications.
    \begin{itemize}
    \item {\bf Mer}: \url{http://merproject.org/}\\
      Fork from the Meego project.\\
      Targeting mobile devices.\\
      Supports x86, ARM and MIPS. \\
      See \url{http://en.wikipedia.org/wiki/Mer\_(software\_distribution)}
    \item {\bf Tizen}: \url{https://www.tizen.org/} \\
      Targeting smartphones, tablets, netbooks, smart TVs and
      In Vehicle Infotainment devices.\\
      Supported by big phone manufacturers and operators \\
      HTML5 base application framework. \\
      Likely to compete against Android! \\
      See \url{http://en.wikipedia.org/wiki/Tizen}
    \end{itemize}
    \column{0.2\textwidth}
    \includegraphics[width=\textwidth]{slides/sysdev-embedded-linux/mer.png}\\
    \includegraphics[width=\textwidth]{slides/sysdev-embedded-linux/tizen.png}\\
  \end{columns}
\end{frame}

\begin{frame}
  \frametitle{System emulator: QEMU}
  \url{http://qemu.org}\\
  Fast processor emulator
  \begin{itemize}
  \item Useful to develop root filesystems and applications when the
    hardware is not available yet (binary compatible development
    boards are also possible).
  \item Emulates the processor and various peripherals (“System
    emulation): x86, ppc, arm, sparc, mips, m68k...
  \item Many different ARM boards supported:\\
    \code{qemu-system-arm -M ?}
  \item QEMU also allows to just run binaries for other architectures
    (``User emulation''). Used inside some building tools.
  \end{itemize}
\end{frame}

\setuplabframe
{OpenWrt}
{
  \begin{itemize}
  \item Build the toolchain with OpenWrt.
  \item See how a build system does everything for you in a one-liner!
  \end{itemize}
}

